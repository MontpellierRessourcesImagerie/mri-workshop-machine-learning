\chapter{Introduction to Machine Learning}

\begin{displayquote}
''Machine learning algorithms build a mathematical model of sample data, known as "training data", in order to make predictions or decisions without being explicitly programmed to perform the task.''\cite{wiki:machine_learning_2019}
\end{displayquote}

Machine learning is a sub--field of artificial intelligence. It is closely related to statistics, statistical learning, data--mining and optimization. A number of machine learning applications are today in use on a regular basis. Machine learning is for example used in 

\begin{itemize}
\item speech recognition
\item image recognition 
\item spam filtering
\item medical diagnosis
\item games: chess, checkers, go
\item price prediction
\item recommendation of products
\item self-driving cars 
\item fraud-detection
\end{itemize}

In bio--image analysis classical machine learning (as opposed to deep-learning) is for example used for:

\begin{itemize}
\item pixel classification (segmentation)
\item cell counting
\item object classification (types of cells)
\item tracking
\item Interactive 3D Segmentation (carving)
\item Boundary-based segmentation with Multicut
\end{itemize}

Deep learning is a special form of machine learning that we will treat in the second part of this course. More recently deep--learning is applied to solve bio-image analysis problems. Besides applications already listed for classical machine learning, some examples are:

\begin{itemize}
\item image restoration
\item prediction of distance maps
\item predict high-resolution images from a low-resolution images in SRLM
\item segmentation of bacteria cells 
\end{itemize}

